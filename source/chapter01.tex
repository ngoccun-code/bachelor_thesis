% !TeX spellcheck = en_US
\chapter{Introduction}%

\section{AUTOTech.agil Project}

This thesis is a part of the project AUTOTech.agil, which is funded by the federal ministry of transport and digital infrastructure under the initiative “Digital Test Beds for Autonomous Driving”. The project is equipped with 7 sensor stations and 75 sensors along the A9-Highway, with a goal of providing solutions and recommendations to improve traffic safety, efficiency and comfort.

\hl{

- the Providentia++ project. The Federal Ministry has funded the project for Digital and Transport since 2017. Since 2020, it has been led by the Chair of Robotics, Artificial Intelligence, and Real-time Systems at the Technical University of Munich’s Department of Informatics.
- Initially, two measurement stations with cameras and Radio Detection and Ranging (Radar) sensors were installed on a test section of the A9 highway near Garching. The sensor data was used to identify vehicles using artificial intelligence and then fused. The resulting digital twin can be used by autonomous cars to make decisions based on events that could not be perceived solely by the vehicle’s onboard sensors.
- The second phase of the project started in 2020. Light Detection and Ranging (LiDAR) sensors were added to the existing perception methods. In addition, the test stretch was expanded into the residential area, which allows surveying intersections, traffic circles, bus stations, and other urban situations.

}


\section{Motivation}
\hl{TODO}

\section{Contributions}
\hl{TODO}
- Our contributions can be summarized as:
- The contributions of this work are summarized as follows:
- In summary, the main contributions of this work are:
- This work presents the following research contributions:

- Extend annotation. 

\section{Structure of the Thesis}
\hl{TODO}

% !TeX spellcheck = en_US
\chapter{Evaluation / Experimentsk}%
- In order to evaluate the performance 
- we conduct experiments on 
- All experiments are conducted on a single 2080Ti GPU card
- For 512×512 images, E2EC achieved a 36 fps inference speed on an NVIDIA A6000 GPU

\section{Evaluation Metrics}

- 2D object detection / segmentation:  Average Precision (AP) 
- 3D object detection (Providentia Mono3D): also  Average Precision (AP) 



- We choose the mean Intersection over Union (mIoU) and mean Average Precision (mAP) as metrics for performance evaluation
- For evaluation, we adopt standard metrics as in most amodal segmentation literature , namely mean average precision (AP). Furthermore, We use mean-IoU to measure the quality of predicted masks.
- In this paper, the mask quality is evaluated in terms of the standard AP metric.

\section{Quantitative results}


\section{Qualitative results}

\section{Inference Speed}

- 1-2 sentences: what is TensorRT
- Pytorch model → tensorRT

- Model performance on RTX-3090 GPU
- Yolov7-640 models achieve frame-rates between 55 and 60 FPS on the RTX-3090 GPU
- Yolov7-1280 achieves 22 FPS, and Yolov7-1920 reaches only 12 FPS
- how many \% speed up? TensorRT vs. .pt


- problems with detecting the instance masks of large objects near the camera.